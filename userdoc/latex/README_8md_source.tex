\hypertarget{README_8md_source}{\section{R\-E\-A\-D\-M\-E.\-md}
}

\begin{DoxyCode}
00001 Bad-Philophobia
00002 ===============
00003 
00004 Auteur
00005 ------
00006 
00007 L\textcolor{stringliteral}{'auteur de ce projet est Rémi NICOLE, étudiant à l'}ESIEE en E<sub>1</sub>-4,
00008 groupe de projet 4M
00009 
00010 Thème
00011 -----
00012 
00013 Le thème du projet est \textcolor{stringliteral}{"Dans une forêt, un homme doit mettre un robot hors}
00014 \textcolor{stringliteral}{service pour s'en délivrer."}.
00015 
00016 
00017 Résumé du scénario
00018 ------------------
00019 
00020 Un homme se retrouve dans une forêt, mystérieusement lié avec un robot
00021 ayant un complexe de pouvoir et doit s\textcolor{stringliteral}{'en débarrasser. Pour se faire, il va}
00022 \textcolor{stringliteral}{devoir ramasser une boule de neige, de laquelle le robot va s'}enticher.
00023 Mais à cause de ces circuits, le dit robot va faire fondre la boule de neige.
00024 Ensuite, le joueur va devoir ramasser une deuxième boule de neige et la lancer
00025 dans l\textcolor{stringliteral}{'eau afin que le robot essaye d'}aller la chercher. À cause du fait que
00026 au contact de l\textcolor{stringliteral}{'eau, les résistances des composants électroniques du robot}
00027 \textcolor{stringliteral}{vont gravement chuter, le robot va donc se retrouver hors service et le joueur}
00028 \textcolor{stringliteral}{a gagné.}
\end{DoxyCode}
